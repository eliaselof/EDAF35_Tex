\section{Module 6 - File System}
\subsection{Red Box}
\begin{itemize}
    \item You should understand these two different access methods, {\bf sequential access} and {\bf direct access}.  (13.2.1 and 13.2.2 in Operating System Concepts)
    \item You should understand what directories are and how they make it possible to organise and access files, i.e, single-level, two-level, tree-structured, acyclic-graph and general-graph. (13.3 in
    Operating System Concepts)
    \item This structure is discussed in the textbook but not very clearly. Try and understand it, but we will not ask
    questions discussing it directly (Section 14.1 in Operating System Concepts).
    \item Need to that know that files are represented as blocks and what the FCB/Inode is (Section 14.1 in Operating
    System Concepts).
    \item Need to understand what these two tables (per-process open-file table and per-process open-file table) are and how calls like open() and read() use and update this table
    (Section 14.2.2 in Operating System Concepts)
    \item Understand these three different allocation methods (contiguous allocation, linked allocation and indexed allocation) and their relative advantages and disadvantages (Section
    14.4 in Operating System Concepts)
    \item Be able to calculate maximum file size a scheme like this can store (indexed allocation) (Section 14.4.3 in Operating System Concepts)
    \item Understand these two algorithms (bit vector and linked list) and their advantages and disadvantages (Section 14.5.1 and 14.5.2 in
    Operating System Concepts)
\end{itemize}



\subsection{Sequential and Direct Access}

\paragraph{Sequential Access}
\begin{itemize}
    \item Data is read or written in a sequential order, starting from the beginning of the file.
    \item It is efficient for tasks like reading logs or processing data in a linear fashion.
    \item Examples: Tape drives, streaming data.
\end{itemize}

\paragraph{Direct Access (Random Access)}
\begin{itemize}
    \item Data can be accessed at any location within the file without reading previous data.
    \item Efficient for tasks requiring quick access to specific data points.
    \item Examples: Hard drives, databases.
\end{itemize}



\subsection{Directory Structures}

\paragraph{What are Directories?}
Directories are used to organize and manage files within a file system. They allow for efficient file storage, access, and hierarchical organization.

\paragraph{Directory Structures}
Different directory structures offer various ways to organize files:

\begin{itemize}
    \item \textbf{Single-Level Directory:}
    \begin{itemize}
        \item All files are stored in a single directory.
        \item Simple but inefficient for large numbers of files.
    \end{itemize}
    
    \item \textbf{Two-Level Directory:}
    \begin{itemize}
        \item A separate directory is created for each user, but all files within each user's directory are stored at the same level.
        \item Offers basic separation of user data.
    \end{itemize}
    
    \item \textbf{Tree-Structured Directory:}
    \begin{itemize}
        \item Directories can contain both files and subdirectories, forming a hierarchical tree.
        \item Most common structure, allows for efficient file organization.
    \end{itemize}
    
    \item \textbf{Acyclic-Graph Directory:}
    \begin{itemize}
        \item Allows directories to contain links to other directories, but no cycles (no directory can be its own ancestor).
        \item Supports shared directories and files.
    \end{itemize}
    
    \item \textbf{General-Graph Directory:}
    \begin{itemize}
        \item Allows directories to contain cycles (directories can be linked to themselves or other directories).
        \item More flexible but can lead to complex management and potential cycles.
    \end{itemize}
\end{itemize}


\subsection{Files and Metadata}

\paragraph{File Representation as Blocks}
Files are stored in the file system as a collection of blocks. A block is a fixed-size unit of data storage, typically ranging from 512 bytes to several kilobytes, used to store the contents of files on a disk.

\paragraph{FCB (File Control Block)}
\begin{itemize}
    \item The FCB is a data structure used to store metadata about a file in a system that supports traditional file systems.
    \item It contains information such as file name, file size, file location, and access permissions.
    \item Primarily used in older file systems like FAT (File Allocation Table).
\end{itemize}

\paragraph{Inode}
\begin{itemize}
    \item The inode is a data structure used in Unix-based file systems to store metadata about a file.
    \item It contains information such as file type, file size, ownership, access permissions, and pointers to the data blocks that store the file contents.
    \item Inodes provide a more efficient method of managing files in a system with many files.
\end{itemize}


\subsection{Open-File Tables}

\paragraph{Per-Process Open-File Table}
\begin{itemize}
    \item The per-process open-file table is a table maintained by the operating system for each process.
    \item It contains information about the files that the process has opened, such as the file descriptor, file offset, and access mode (read/write).
    \item When a process opens a file using the \texttt{open()} system call, a new entry is added to this table.
\end{itemize}

\paragraph{System-Wide Open-File Table}
\begin{itemize}
    \item The system-wide open-file table is a global table maintained by the operating system.
    \item It keeps track of all open files across all processes, storing information like the file location and the number of file descriptors pointing to the file.
    \item When a process opens a file, the operating system checks this table to see if the file is already open and, if so, updates the reference count.
\end{itemize}

\paragraph{File Operations (open(), read())}
\begin{itemize}
    \item \texttt{open()}: When a file is opened, the operating system updates the per-process open-file table with an entry for the file. If the file is already open, it retrieves the file descriptor from the system-wide table and adds it to the process's table.
    \item \texttt{read()}: When a process reads a file, the operating system uses the file descriptor from the per-process open-file table to find the file’s entry in the system-wide table. The file offset and other information are used to read data from the file.
\end{itemize}


\subsection*{Types of File Allocation Methods}

\paragraph{Contiguous Allocation}
\begin{itemize}
    \item Files are stored in contiguous blocks on the disk.
    \item Advantages:
    \begin{itemize}
        \item Fast access to files due to sequential storage.
        \item Simple to implement.
    \end{itemize}
    \item Disadvantages:
    \begin{itemize}
        \item External fragmentation as files grow or shrink.
        \item Difficult to allocate space for large files if contiguous space is unavailable.
    \end{itemize}
\end{itemize}

\paragraph{Linked Allocation}
\begin{itemize}
    \item Files are stored in scattered blocks, with each block pointing to the next.
    \item Advantages:
    \begin{itemize}
        \item Eliminates external fragmentation.
        \item Easy to expand files without needing contiguous space.
    \end{itemize}
    \item Disadvantages:
    \begin{itemize}
        \item Slower access due to pointer traversal.
        \item Additional overhead for storing pointers in each block.
    \end{itemize}
\end{itemize}

\paragraph{Indexed Allocation}
\begin{itemize}
    \item A table (index) is used to keep track of the block locations for each file.
    \item Advantages:
    \begin{itemize}
        \item Eliminates fragmentation (both internal and external).
        \item Efficient access as all blocks can be accessed directly using the index.
    \end{itemize}
    \item Disadvantages:
    \begin{itemize}
        \item Requires extra space for the index table.
        \item Can lead to overhead when the index table is large or needs to be stored on disk.
    \end{itemize}
\end{itemize}


\subsection{Indexed Allocation and Maximum File Size}

\paragraph{Indexed Allocation}
In indexed allocation, each file has an index block that stores pointers to the data blocks that hold the actual file data. The maximum file size a scheme using indexed allocation can store depends on the size of the index block and the size of the data blocks.

\paragraph{Formula for Maximum File Size}
\begin{itemize}
    \item Let:
    \begin{itemize}
        \item $B_d$ = size of each data block.
        \item $B_i$ = size of the index block.
        \item $P$ = size of a pointer (usually in bytes, e.g., 4 bytes for 32-bit pointers).
    \end{itemize}
    \item The index block can store $\frac{B_i}{P}$ pointers.
    \item Therefore, the maximum number of data blocks a file can occupy is $\frac{B_i}{P}$.
    \item Thus, the maximum file size $S_{max}$ is given by:
    \[
    S_{max} = \left( \frac{B_i}{P} \right) \times B_d
    \]
\end{itemize}

\paragraph{Example Calculation}
Suppose:
\begin{itemize}
    \item Each data block is 4 KB ($B_d = 4096$ bytes).
    \item The index block is 1 KB ($B_i = 1024$ bytes).
    \item Each pointer is 4 bytes ($P = 4$ bytes).
\end{itemize}

Using the formula:
\[
S_{max} = \left( \frac{1024}{4} \right) \times 4096 = 256 \times 4096 = 1,048,576 \text{ bytes} = 1 \text{ MB}
\]
Thus, the maximum file size that can be stored in this indexed allocation scheme is 1 MB.



\subsection{Bit Vector and Linked List Algorithms}

\paragraph{Bit Vector Allocation}
\begin{itemize}
    \item The bit vector (or bitmap) is a simple data structure used to manage free disk blocks.
    \item A bit is allocated for each disk block. A value of \texttt{0} indicates that the block is free, while \texttt{1} indicates that the block is in use.
    \item Advantages:
    \begin{itemize}
        \item Simple and efficient for managing free space.
        \item Fast to locate a free block, especially in systems with contiguous allocation.
    \end{itemize}
    \item Disadvantages:
    \begin{itemize}
        \item Requires a large amount of memory to store the bit vector, especially for large disk sizes.
        \item Can be inefficient for sparse disk usage, as it requires a bit for every block.
    \end{itemize}
\end{itemize}

\paragraph{Linked List Allocation}
\begin{itemize}
    \item The linked list algorithm maintains a linked list of free blocks. Each block contains a pointer to the next free block.
    \item Advantages:
    \begin{itemize}
        \item No memory overhead for the bitmap, and it efficiently uses space for sparse allocations.
        \item Can easily track fragmented free space.
    \end{itemize}
    \item Disadvantages:
    \begin{itemize}
        \item Slower to locate a free block compared to bit vector because it requires traversing the list.
        \item Extra storage overhead for maintaining the pointers in each free block.
    \end{itemize}
\end{itemize}